\documentclass[preprint2]{aastex}
\usepackage{natbib}
\usepackage{graphicx}
\usepackage{amsmath}
\usepackage{amssymb}
\usepackage{epstopdf}
\usepackage{natbib}
\usepackage[xindy, toc, hyperfirst=false, nolist, nostyles, sanitize={name=false,description=false,symbol=false}]{glossaries}
\glsdisablehyper
%\usepackage[hyperref,x11names, table]{xcolor}



%graphic rules

\DeclareGraphicsRule{.tif}{png}{.png}{`convert #1 `dirname #1`/`basename #1 .tif`.png}

% Paths for tables and plots - normally are in subdirectories, but for submission can be in the main dir
\newcommand{\plotdir}{plots}
\newcommand{\tabledir}{tables}


%define datasets used



%inputting the glossary
\input{glossary.tex}


\shortauthors{W.E. Kerzendorf et al.}
\shorttitle{SN1604 companion search}

\begin{document}

% TITLE AND AUTHORS
%-----------
\title{A reconnaissance of the possible donor stars to the Kepler supernova}

\author{Wolfgang~E.~Kerzendorf\altaffilmark{1}\altaffilmark{2}, Michael Childress\altaffilmark{1}, Julia Scharw\"{a}chter\altaffilmark{3}, Tuan Do\altaffilmark{2}, Brian~P.~Schmidt\altaffilmark{1}} 
\email{wkerzend@mso.anu.edu.au}


\altaffiltext{1}{Research School of Astronomy and
Astrophysics, Mount Stromlo Observatory, Cotter Road, Weston Creek,
ACT 2611, Australia}

\altaffiltext{2}{Department of Astronomy and Astrophysics, University of Toronto, 50 Saint George Street, Toronto, ON M5S 3H4, Canada}


\altaffiltext{3}{Observatoire de Paris, LERMA (CNRS: UMR 8112), 61 Av. de l'Observatoire, 75014, Paris, France}

\begin{abstract}
Type Ia supernovae are not only one of the main drivers of chemical evolution in the Universe, but are also unrivalled distance probes. This makes them a key topic in modern astrophysics. Type Ia supernovae are the explosion of a massive ($> 1 \msun$) CO white dwarf, which are not created in large enough numbers as part of a normal single stellar evolution and thus require binary stellar evolution. The community proposes two scenarios, either accreting matter (in a stable manner) from a companion star or the merger of two less massive white dwarfs. Which one of these scenarios produces Type Ia supernovae remains an unanswered question. Only the accretion scenario provides a, potentially directly observable, surviving companion and thus proof. 
In this paper we scrutinize 24 central stars  of the SN~1604 remnant to search for a possible surviving companion star. We conclude that all candidates are consistent with unrelated stars and none exhibit properties uniquely consistent with a companion. Finally, we can rule out a red giant companion star (suggested previously for \sn{1604}) with high certainty (the brightest star is V=16.5, a red giant scenario requires a star with V$\approx 12$).
\end{abstract}

\maketitle
\section{Introduction}

\sneia\ provide a means to measure cosmological distances and are also major contributors to the chemical evolution of the Universe, enriching the cosmos with large amounts of iron-peak elements. Despite considerable effort to identify the systems that become SN Ia over the past decade, no consensus has yet emerged. It is widely accepted that \sneia\ are the thermonuclear explosions of a massive \gls{cowd} . The community suggests two main progenitor scenarios \cite[for a review see][and references therein]{2012NewAR..56..122W}. In the first scenario, the WD accretes mass from a non-degenerate companion (known as a donor star) until the central temperature and density exceed the cooling threshold and a run-away nuclear burning occurs. This is usually described as the \gls{sds}. In the second scenario, two WDs merge, leading to a cataclysmic explosion (\gls{dds}).

Both events make a number of predictions, however only the \gls{sds} offers a directly observable prediction - the remaining donor star. On the face of it, the \gls{sds} looks easily falsifiable. The donor star should remain \citep[e.g.][]{2000ApJS..128..615M} and be easy to detect in historical remnants. There has been considerable effort \citep{2004Natur.431.1069R, 2009ApJ...691....1G,2009ApJ...701.1665K,2012Natur.481..164S,2012Natur.489..533G,2012arXiv1210.2713K,2012ApJ...759....7K, 2012ApJ...747L..19E} to find the donor stars of ancient supernovae. Two remnants in the LMC and \sn{1006}\ have proven to be void of progenitors to a relatively stringent detection limit. A possible progenitor has been found in SN 1572 \citep{2004Natur.431.1069R}, although the validity of this connection has been challenged \citep{2012arXiv1210.2713K}. Despite, these negative results this does not necessarily disprove the validity of the \gls{sds}, as one can construct scenarios for the single-degenerate case, which produce surviving companions that do not stand out among their neighbouring stars \citep[e.g.][]{2012ApJ...760...21P, 2013arXiv1303.2691L, 2003astro.ph..3660P}. 

The recent discoveries of \gls{csm} around roughly a quarter of normal \sneia\ \citep{2007Sci...317..924P, 2009ApJ...702.1157S, 2011Sci...333..856S, 2012ApJ...752..101F} seemingly suggests two subtypes. As \gls{csm} interaction is favoured by the \gls{sds} and is rarer, the lack of direct detection of a donor star could just be the relative rarity of these objects.

In this work we turn to the last of the young Galactic supernova remnants - the remnant of \sn{1604}\ (also known as Kepler's \sn). There's a dispute about the precise distance \citep[$4-6.4$\,\kpc; see][and references therein]{2012A&A...537A.139C}, and we choose to adopt the upper and more conservative limit of 6.4~\kpc\ \citep{1999AJ....118..926R} for this work. Furthermore, there's a known maximum apparent brightness \citep[V=-3;][]{1971SvA....14..798P}, which, together with a usual absolute brightness of \sneia \citep[$M_V=19.3$;][]{2011ApJ...732..129R} and an extinction of $A_V=2.8$ leads to a distance of $\approx 5~\kpc$, consistent with the other measurements. In the case of V-band observations, the SN's observed brightness provides an distance / extinction independent estimate of the distance modulus, useful for estimating the luminosity of stars if they are associated with the SN remnant. The relatively high extinction in the direction of the remnant \citep[$A_V=2.8$, ][]{2007ApJ...668L.135R}, makes stellar photometry and spectroscopy difficult. Compared to its Galactic siblings (\sn{1006}\ and \sn{1572}) the Kepler \snr\ shows a very unusual structure and the presence of a nitrogen-rich shell.  \citet{2012A&A...537A.139C} attributes the shell to a progenitor with high mass loss prior to explosion - a calling card for the \gls{sds}. This large amount of interstellar matter suggests that Kepler might have belonged to the class of \sneia\ showing \gls{csm}\ interaction in their spectra. Finally, the remnant shows a high systemic velocity of  $250~\kms$ \citep{1991ApJ...366..484B,2003A&A...407..249S}. This high systemic velocity, including a radial velocity component of -180~\kms , should help make it possible to more easily identify a potential donor star, as any star associated with the SN should have -180~\kms of motion added to its orbital motion at the time of the explosion.



Here, we report a photometric and spectroscopic search for a donor star in the \sn{1604}\ remnant. In Section~\ref{sec:observations} we give an overview of the observation of the spectroscopic data and describe the \gls{hst} data. We describe the analysis of the spectroscopic data in Section~\ref{sec:analysis} and discuss its implications in Section~\ref{sec:discussion}. We conclude this paper and discuss possible future work in Section~\ref{sec:conclusion}.


\section{Observations}
\label{sec:observations}
\subsection{Integral Field Spectroscopy}
We used the \gls{ifu} called \gls{wifes} mounted on the ANU 2.3m Telescope at the Siding Spring Observatory  to undertake the spectroscopic observations in this study.. The \gls{wifes}-spectrograph is an image slicer with twenty five $38\times 1\arcsec$ slitlets and 0.5\arcsec\ sampling in the spatial direction on the detector. We chose this instrument for its large field of view ($25\arcsec\times38\arcsec$) and a resolution of $R=7000$ required for adequate radial velocity measurements. 

We have adopted \rasc{17}{30}{41}{6} \decl{-21}{29}{17} as the centre of the remnant from \citet{2008ApJ...689..225K}. Due to the potential uncertainty in the centre determination \citep[for a more detailed discussion about the difficulty of centre determination see Section~2.2 in][]{2012ApJ...759....7K} we split up our observations to cover two fields of $25\arcsec\times38\arcsec$ (see Figure~\ref{fig:sn1604_overlay}). This setup gives us a minimum search distance of $19\arcsec$ from the proposed centre, corresponding to a velocity of $1420~\kms$ (in the plane of the sky) assuming a remnant distance of $6.4~\kpc$ \citep{2012A&A...537A.139C}. 

% !TEX root = sn1604_wifes.tex

\begin{figure*}[tb!]
   \centering
   \includegraphics[width=\textwidth]{\plotdir /sn1604_overlay.pdf} 
   \caption{SN1604 \gls{2mass} image with contours of ACIS X-Ray image \citep[ObsID 116; ][]{2002ApJ...581.1101H}. In this case, 19\arcsec\ equals 1420 \kms\ at 6.4~\kpc over 400 years. The dashed circle describes a 60\arcsec circle where we photometrically searched for potential companion stars.}
   \label{fig:sn1604_overlay}
\end{figure*}


We selected the I7000 grating which covers a spectral region  ($6830~-~9120~\AA$) with 0.2\AA (7~\kms) resolution. This region includes the \ion{Ca}{2} Triplet (8498\AA, 8542\AA, and 8662\AA) and maximises the Signal to Noise of the stars observed, which are heavily reddened by dust. The \ion{Ca}{2} provides good radial velocity measurements even in low \gls{signalnoise} ratio observations.

The observations were carried out  on 2010 June 15 and 2010 June 16. The exposure time for all observations was $600~\textrm{s}$ with the southern field being observed 33 times ($5.5~h$) and the northern field observed 19 times ($3.2~h$) as weather conditions deteriorated. 
The seeing ranged between 1.7\arcsec and 2.7\arcsec with a median of 1.9\arcsec (see Figure~\ref{fig:sn1604_candidates} left panel for a sample of the data quality). 

%% !TEX root = sn1604_wifes.tex

\begin{figure*}[tb!]
   \centering
   \includegraphics[width=\textwidth]{\plotdir /wifes_data.pdf} 

   \caption{An overview of the median of a data cube (in this case the southern field)}
   \label{fig:sn1604_wifes_data}
\end{figure*}




% !TEX root = sn1604_wifes.tex

\begin{figure*}[tb!]
   \centering
   \includegraphics[width=.49\textwidth, bb = 0 0 500 400,clip]{\plotdir /sn1604_data_candidates.pdf} 
   \includegraphics[width=.49\textwidth, bb = 0 0 500 400,clip]{\plotdir /sn1604_hst_candidates.pdf}
   \caption{\textbf{Left panel:} The combination of a median (in the spectral direction) of two of our spectral cubes. The extraction regions marked in blue. The seeing is comparable to the 2MASS, enabling a direct comparison of 2MASS magnitudes and the spectra. Due to dithering, some of the stars are not on these cubes.\newline \textbf{Right~panel:} HST F550M image of the central of region of SN1604. We have marked the extraction regions with blue circles and the individual sources greater than $V=19.21$ ($10\lsun$ at the distance of the remnant taking the difference between F550M and V-filter into account) with red circles. There are a few extraction regions that contain more than one bright star .}
   \label{fig:sn1604_candidates}
\end{figure*}


%\begin{figure*}[tb!]
 %  \centering
 %  \includegraphics[width=\textwidth]{\plotdir /sn1604_hst_candidates.pdf}
 %  \caption{}
%   \label{fig:sn1604_hst_candidates}
%\end{figure*}




WiFeS data were processed using the PyWiFeS package\footnote[1]{{\tt
http://www.mso.anu.edu.au/pywifes/}}.  Wavelength solutions for each
data cube were derived from night sky emission lines, achieving a
residual wavelength solution scatter of 0.1\AA.


In a final step, a simple spatial world coordinate system was applied to each cube with the help of \gls{2mass} for object coordinates (as \gls{2mass} shows a similarly broad PSF).



\subsection{Photometry}

To obtain accurate photometry, we used archival \gls{hst} data (\gls{hst} program GO-9731) observed in August 2003. The data consist of two F550M images with an exposure time of $240\,\textrm{s}$ each, observed with the \gls{acs}. We analyzed the frames using \gls{dolphot} and extracted magnitudes for all stars and corrected it to the V-Band (difference in flux between F550M and V for Vega is 0.12 magnitudes). Table \ref{tab:hst_candidates} shows the magnitudes of stars with more than 10 solar luminosities in the V-Band at the distance of the remnant assuming an extinction of $A_V = 2.8$ \citep{2007ApJ...668L.135R}. In addition, we have supplemented the HST photometry of these objects with photometric information from the \gls{nomad}. Furthermore, Figure~\ref{fig:sn1604_candidates}  (right panel) shows the possible companion candidates and an overlay of the \gls{wifes}-fields onto the \gls{hst} image. 


\begin{deluxetable}{lcccccccccccc}
\rotate
\tablecaption{HST measurements of candidates \label{tab:hst_candidates}}
\tablehead{\colhead{Name} & \colhead{RA} & \colhead{Dec} & \colhead{F550M} & \colhead{Luminosity\tablenotemark{a}}
& \colhead{NOMAD ID} & \colhead{B} & \colhead{V} & \colhead{R} & \colhead{J} & \colhead{H} & \colhead{K}\\
\colhead{-} & \colhead{hh:mm:ss.ss} & \colhead{dd:mm:ss.s} & \colhead{mag} & \colhead{$L_\odot$} &  \colhead{-}
& \colhead{mag} & \colhead{mag} & \colhead{mag} & \colhead{mag} & \colhead{mag} & \colhead{mag}}
\startdata
A1 & 17:30:41.88 & -21:29:17.3 & 18.93 & 15 & 0685-0474466 & -- & -- & 17.6 & 15.1 & 14.5 & 14.2\\ 
B1 & 17:30:41.13 & -21:29:14.4 & 17.55 & 53 & 0685-0474416\tablenotemark{b} & 18.0 & 16.7 & 13.9 & 14.0 & 13.2 & 12.9\\ 
B2 & 17:30:41.24 & -21:29:15.7 & 18.98 & 14 & 0685-0474416\tablenotemark{b} & 18.0 & 16.7 & 13.9 & 14.0 & 13.2 & 12.9\\ 
C1 & 17:30:41.75 & -21:29:26.4 & 18.90 & 15 & 0685-0474460 & -- & -- & 17.9 & 15.6 & 14.9 & 14.6\\ 
D1 & 17:30:41.04 & -21:29:11.5 & 18.07 & 33 & 0685-0474416 & 18.0 & 16.7 & 13.9 & 14.0 & 13.2 & 12.9\\ 
E1 & 17:30:41.48 & -21:29:06.6 & 17.24 & 71 & 0685-0474437 & 17.1 & 16.1 & 15.3 & 13.6 & 12.9 & 12.6\\ 
E2 & 17:30:41.47 & -21:29:07.3 & 17.79 & 43 & 0685-0474437 & 17.1 & 16.1 & 15.3 & 13.6 & 12.9 & 12.6\\ 
E3 & 17:30:41.26 & -21:29:07.2 & 18.95 & 15 & 0685-0474437 & 17.1 & 16.1 & 15.3 & 13.6 & 12.9 & 12.6\\ 
F1 & 17:30:40.95 & -21:29:21.5 & 17.95 & 37 & 0685-0474398 & 18.8 & -- & 15.9 & 14.4 & 13.7 & 13.4\\ 
G1 & 17:30:41.01 & -21:29:26.0 & 18.92 & 15 & 0685-0474409 & -- & -- & -- & 15.5 & 14.6 & 14.3\\ 
H1 & 17:30:40.76 & -21:29:17.0 & 18.69 & 18 & 0685-0474388 & -- & -- & -- & 15.0 & 14.2 & 14.0\\ 
H2 & 17:30:40.60 & -21:29:17.4 & 19.16 & 12 & 0685-0474388 & -- & -- & -- & 15.0 & 14.2 & 14.0\\ 
I1 & 17:30:41.23 & -21:29:30.1 & 19.18 & 12 & 0685-0474414 & 20.3 & -- & 17.1 & -- & -- & --\\ 
J1 & 17:30:42.70 & -21:29:14.2 & 19.14 & 12 & 0685-0474517 & 19.0 & -- & 17.8 & 15.4 & 14.8 & 14.4\\ 
K1 & 17:30:40.39 & -21:29:15.6 & 17.91 & 38 & 0685-0474365 & -- & 16.4 & -- & 14.8 & 14.2 & 14.1\\ 
K2 & 17:30:40.56 & -21:29:13.9 & 19.24 & 11 & 0685-0474365 & -- & 16.4 & -- & 14.8 & 14.2 & 14.1\\ 
L1 & 17:30:40.67 & -21:29:27.6 & 16.50 & 140 & 0685-0474384 & 17.4 & 15.9 & 12.8 & 13.1 & 12.3 & 12.0\\ 
M1 & 17:30:42.14 & -21:29:01.4 & 19.16 & 12 & 0685-0474479 & -- & -- & 18.0 & 15.8 & 15.2 & 14.8\\ 
N1 & 17:30:40.27 & -21:29:19.3 & 17.38 & 62 & 0685-0474362 & 18.5 & 16.2 & -- & 14.0 & 13.2 & 13.0\\ 
O1 & 17:30:42.71 & -21:29:31.9 & 17.94 & 37 & 0685-0474522 & 18.7 & 17.7 & 16.7 & 14.7 & 13.9 & 13.6\\ 
P1 & 17:30:40.62 & -21:29:34.2 & 17.44 & 59 & 0685-0474383 & 18.1 & 16.3 & -- & 13.7 & 13.0 & 12.7\\ 
P2 & 17:30:40.78 & -21:29:34.7 & 17.89 & 39 & 0685-0474383 & 18.1 & 16.3 & -- & 13.7 & 13.0 & 12.7\\ 
Q1 & 17:30:42.88 & -21:29:03.2 & 18.55 & 21 & 0685-0474537 & -- & -- & 17.8 & 15.8 & 14.9 & 14.9\\ 
R1 & 17:30:42.54 & -21:29:36.6 & 18.77 & 17 & 0685-0474505 & -- & -- & 17.2 & 15.3 & 14.5 & 14.4\\ 

\enddata
\tablenotetext{a}{Assuming the candidate to be at the distance of SN1604}
\tablenotetext{b}{Extraction Region B and D are covered by one NOMAD ID}
\end{deluxetable}






\section{Analysis}
\label{sec:analysis}
 

Using the HST images as a template we selected extraction regions in our spectral cubes with an aperture of 2\arcsec, a good compromise between using the available signal and avoiding contamination from close stars. Despite the small aperture, some extraction regions contain two or three stars. The aperture was then summed up over the whole spectral axis. A sky measurement was obtained in a similar fashion using an annulus with an inner radius of 4\arcsec and an outer radius of 6\arcsec. To alleviate contribution of neighbouring stars,  the individual pixels of the sky annuli were median combined during extraction. We scaled the skylines of the sky annuli to the sky lines of the spectra to avoid an over- or under-subtraction to best remove any residual background..

We concentrated on the wavelength region 8400--8700\AA\ for the next part of the analysis - as this contains the strong \ion{Ca}{2} lines which are useful for radial velocity measurements. The sky-frame scale was fit with a minimizer \citep{powell1964efficient} simultaneously with a continuum (third-order polynomial), and a model of the sun (\teff=5780, \logg=4.4, and \feh=0.0). Although the S/N ratio is in principle high enough to measure stellar parameters, the uncertainty in the continuum placement due to sky
subtraction errors makes the parameter estimations unreliable. In addition, we lack colour photometry to aid in the determination of stellar parameters. The unreliable placement of the continuum does not affect the radial velocity measurements as they only rely on the position and not the depth of lines.

The scaled sky was subtracted from the spectra and the resulting spectra (for each extraction site) were interpolated on a common wavelength grid and summed up. Next, we convolved a solar spectrum to the required resolution and shifted these synthetic spectra between -400\kms and 200\kms in a thousand equally spaced steps. We then compared the observed spectra with the set of synthetic spectra (using the root-mean-square technique) and chose the velocity corresponding to the lowest root-mean-square. Subsequently, we compared each fit with the spectrum and ascertained that the \ion{Ca}{2} features were clearly visible in the candidate spectrum and coincided with the shifted solar spectrum (see Figure~\ref{fig:kepler-g}). Furthermore, in the cases where HST photometry showed the presence of two stars, we checked the RMS fit for secondary minima that arises when there's two distinct sets of spectra. Only Kepler-H showed two clear sets of \ion{Ca}{2} features in the spectrum as well as in the RMS fit (henceforth designated as F1 and F2). Finally, we determined the typical error of the radial velocity to be $\approx4.5~\kms$  (or a tenth of a resolution element), by measuring the radial velocity in individual frames of a few candidates and looking at the resulting distribution. 

When compared to model spectra we find that the resolution and quality of the spectra does not allow for a reliable determination of the rotational velocity to better than 200~\kms (see Figure~\ref{fig:kepler-g}). 

% !TEX root = sn1604_wifes.tex

\begin{figure*}[tb!]
   \centering
   \includegraphics[width=\textwidth]{\plotdir /kepler-k-comparison.pdf} 

   \caption{The Kepler-K candidate showing a radial velocity of -155~\kms.  Such a velocity is consistent with that expected of more than a  third of the donor stars of the Kepler-SNR, but also of 5\% of unassociated stars in the direction of the remnant. Given that this study has analysed 20 stars' radial velocities, this velocity is not, on its own, indicative of an unusual star.}
   \label{fig:kepler-g}
\end{figure*}




% !TEX root = sn1604_wifes.tex
\begin{deluxetable}{lccc}
\tablecaption{Radial velocity measurements of candidates \label{tab:hst_vrad}\tablenotemark{a}}

\tablehead{\colhead{Name} & \colhead{\vrad} & \colhead{Distance from center} & \colhead{$P(\textrm{donor} | \vrad)$\tablenotemark{b}}\\
\colhead{Designation} & \colhead{\kms} & \colhead{arcsec} & \colhead{1}}
\startdata
A & -69.07 & 3.75 & 0.05 \\ 
B & 167.17 & 5.99 & 0.00 \\ 
C & 7.01 & 9.59 & 0.00 \\ 
D & -74.27 & 9.72 & 0.06 \\ 
E & -38.64 & 10.29 & 0.02 \\ 
F & -10.51 & 10.33 & 0.01 \\ 
G & -94.29 & 12.34 & 0.10 \\ 
H I\tablenotemark{c} & 177.58 & 12.81 & 0.00 \\ 
H II\tablenotemark{c} & 23.92 & 12.81 & 0.00 \\ 
J & 39.04 & 15.46 & 0.00 \\ 
K & -155.96 & 16.30 & 0.36 \\ 
L & -88.29 & 16.92 & 0.09 \\ 
M & 14.61 & 17.20 & 0.00 \\ 
N & -81.88 & 19.05 & 0.07 \\ 
O & -10.71 & 21.37 & 0.01 \\ 
P & 86.69 & 21.60 & 0.00 \\ 
Q & -59.06 & 22.43 & 0.04 \\ 
R & 41.44 & 23.45 & 0.00 \\ 

\enddata
\tablenotetext{a}{We could not reliable measure a radial velocity for Candidate I.}
\tablenotetext{b}{Probability to be the donor star for the given \vrad\ and using the priors from \gls{besancon} and  \citet{2008ApJ...677L.109H}. We assume in this case that  one of nineteen candidates must be the donor star.}
\tablenotetext{c}{Extraction region H shows to different radial velocities, stemming from the two different stars visible in the HST image.}
\end{deluxetable}


\section{Discussion}
\label{sec:discussion}

The \gls{sds} predicts an escape velocity of the donor of up to 200\,\kms for main sequence stars and less for sub giants or giant stars \citep[down to roughly 60~\,\kms;][]{2008ApJ...677L.109H}. In the case of \sn{1604}, this velocity signature might very easily be lost in the kinematic signature of the Galaxy, although the systemic velocity of the Kepler remnant helps separate candidates from the normal motions of stars along the line of sight. We compare our radial-velocity measurements to theoretical predictions by \citet{2008ApJ...677L.109H} and the \gls{besancon} of galactic dynamics (choosing 1 square degree area around \sn{1604}\ and limiting to magnitudes between $15<$V$<20$). In Figure~\ref{fig:sn1604_besancon} we use a Monte Carlo simulation using the distribution of radial velocities from \citet{2008ApJ...677L.109H} (including main-sequence to giant donors) and a random distribution of ejection angles to obtain a radial velocity distribution. Furthermore, we center this on \snr{1604}'s systemic radial velocity of $-185~\kms$\ \citep{2003A&A...407..249S}. When looking at the \citet{2008ApJ...677L.109H} distribution in Figure~\ref{fig:sn1604_besancon}, half of the possible cases would produce a star being a significant outlier compared to the unrelated background and foreground stars. However, none of our stars are significant outliers to the \gls{besancon}, with our measured radial velocity distribution agreeing very well with the \gls{besancon}. Kepler-K can be seen as the most anomalous star studied. However, without additional information, Kepler-K's velocity does not stand-out from the expected velocity distribution for stars along the line of site to Kepler. If we add a prior that one of the 19 stars is associated with the remnant, then Kepler-K is the most likely star to be associated (36\% chance - or approximately 4 times more likely than the next most likely star, Kepler-G).

\citet{2000ApJS..128..615M} suggests that giant donor stars will be on the order of 1000~\lsun\  for at least 100, 000 years post-explosion. None of the stars in our sample are in that brightness range (the brightest star being Kepler-L with $139~\lsun(V)$ at 6.4~\kpc). Therefore, there seems to be no viable giant donor located in \sn{1604}. This is evidence against the AGB model suggested by \citet{2012A&A...537A.139C} and \citet{2013ApJ...764...63B}, which is one scenario explaining the unusual \gls{csm} that surround the Kepler remnant, with a $4 - 5 \msun$ AGB donor. \citet{2012A&A...537A.139C} suggest a star with $m_v=12.0 \pm 0.5$ which is not present in our candidates. 

\cite{2013arXiv1305.0567W} have suggested that the centre of remnants is not easily determined and suggest widening the progenitor searches. We have used the \gls{nomad} catalog to find bright companions in a 60\arcsec circle around our current centre (see Figure~\ref{fig:sn1604_overlay}). The brightest star in K-Band reaches K=10.5 (distance from centre 43\arcsec)  the brightest star in V-Band reaches 14.7 (distance from centre 40\arcsec), the V-band equating to $330~\lsun$.   Finally, \citet{2012A&A...537A.139C}  suggest the possibility of only having a white dwarf left over ($0.8~\msun$).  A normal white dwarf would be below our detection limit, however, there is no obvious physical mechanism that would a bare WD core 400 years after the SN explosion.

For main-sequence and sub-giant companions two new studies  \citep{2012arXiv1205.5028S,2012ApJ...760...21P} suggest a luminosity of $\approx 20 - 275 \lsun$ for companion stars 400 years after the explosion (see Figure~6 in \citep{2012ApJ...760...21P}). restricting ourselves to objects that have $\ge 1 \%$ probability and $\lsun > 20$, we are left with nine candidates (D1, E1, E2, F1, K1, K2, L1, N1, and O1). These are the most notable stars for further follow-up. In addition, a run-away star can be hidden in the general distribution of radial velocities in the direction of \sn{1604}. This is due, both, the very broad distribution of radial velocities (as seen in Figure~\ref{fig:sn1604_besancon}) and the possibility of a high proper motion coupled with a low radial velocity. Obtaining new HST-ACS images would help isolate any associated star by utilizing all three velocity components. 

Another interpretation, is that there are no main-sequence or sub-giant donors in \sn{1604}. But there are some caveats to any such statement.  In some cases we extract the spectrum of two or more stars due to their close proximity on the sky, but do not find two radial velocities (except in extraction aperture Kepler-H) , meaning that a star's velocity is missing. Possible stars with $L>20~\lsun$ include stars E2 and P2, although with their relative brightnesses compared to the primary star, a large negative radial velocity, consistent with a donor star, should have been visible.
 



% !TEX root = sn1604_wifes.tex

\begin{figure*}[tb!]
   \centering
   \includegraphics[width=\textwidth]{\plotdir /sn1604_besancon.pdf} 

   \caption{Comparison of \gls{besancon} (1 square degree around Kepler, V cut between 15 and 20) and \citet{2008ApJ...677L.109H} and our measured values. The top panel shows the cumulative distribution for the \gls{besancon} and our measured values. The \citet{2008ApJ...677L.109H} distribution are the result of a Montecarlo simulation including the distribution of random angles and two \citet{2008ApJ...677L.109H} distributions}
   \label{fig:sn1604_besancon}
\end{figure*}





\section{Conclusion \& Future Work}

\label{sec:conclusion}

In this work, we present photometric and spectroscopic observations of candidate stars in the centre of the \sn{1604}\ remnant. We can rule out red giant companions due to the lack of bright enough stars in the field (brightest star Kepler-L with V=16.5). In addition, we can rule out many candidates in the field, that have inconsistent radial velocity signatures (all stars with a $\vrad>0~\kms$). Finally, our radial velocity measurements do not identify any peculiar star in this data set beyond what is expect from a Galactic distribution, and are therefore consistent with \snia\ progenitor scenarios which do not leave behind a bright donor star, similar to work on other historical remnants such as those which have scrutinized \sn{1006}\ \citep{2012Natur.489..533G,2012ApJ...759....7K}, and the LMC results \citep{2012ApJ...747L..19E, 2012Natur.481..164S}. 

Both the newly discovered \gls{csm} interacting \sneia\ and the lack of donor stars requires an explanation that neither the traditional \gls{sds} or the traditional \gls{dds} can provide. \citet{2011ApJ...730L..34J,2011ApJ...738L...1D,2012ApJ...744...69H,2012ApJ...756L...4H} have suggested a modified \gls{sds} in which the companion has time to evolve and become a white dwarf before the explosion of the primary or is intrinsically dim \citep{2012ApJ...758..123W}. This can explain the difficulty finding said companion. If the white dwarfs merge in a common envelope this might explain the \gls{csm} interaction that is normally not expected for a \gls{dds} and also explains the lack of a companion \citep[][, van Kerkwijk priv. comm. ]{2011MNRAS.417.1466K}.  

To further test main-sequence and sub-giant progenitor scenarios, additional observations are called for. High-resolution spectra of the viable stars with $L>20\lsun$ (D1,E1,E2,F1,K1,K2,L1,N1,O1) could be made with a multi-object fibre system such as VLT+FLAMES. Such observations would be able to both detect rotation and better detect the blended stars kinematic signatures.In addition, a single ACS HST follow-up observation would enable an accurate measurement of all of the remnant's stars' proper motions, enabling detection of objects with motions consistent with the remnant down to the luminosities expected for even a hot white dwarf. 

Currently only three remnants have been spectroscopically searched for donor stars (SN1572, SN1006 and SN1604), which is a statistically low sample. The fourth known Type Ia \snr RCW86 lends itself for an equally involved donor star search. At first glance, this remnant with its low extinction \citep[$A_V \approx 1.7$;][]{1983MNRAS.204..273L} and close proximity to earth \citep[$d=2.5~\kpc$; ][]{2011ApJ...741...96W})  makes it seem ideal. However, there are two seemingly separate expansion fronts (one in the south west and one in the north east) that would make a center determination very difficult. A safe approach is a search area that encompasses both centers, however this would involve scrutinizing many stars. This makes RCW86 currently not feasible for a search and thus Kepler is the last remnant easily searched in the Galaxy. In the future, GAIA will measure distances and thus be able to narrow down the possible candidate stars for RCW86.



\section{Acknowledgements}

We like to thank Ben Shappee for useful discussions on post explosion donor evolution. Furthermore, Steve Reynolds, Kazik Borkowski, Mary Burkey, and Jacco Vink were providing crucial points about a possible AGB star. 

\bibliographystyle{hapj}
\bibliography{sn1604_wifes}



\end{document}
