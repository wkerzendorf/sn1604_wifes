% !TEX root = sn1604_wifes.tex
\begin{deluxetable}{lccc}
\tablecaption{Radial velocity measurements of candidates \label{tab:hst_vrad}\tablenotemark{a}}

\tablehead{\colhead{Name} & \colhead{\vrad} & \colhead{Distance from center} & \colhead{$P(\textrm{donor} | \vrad)$\tablenotemark{b}}\\
\colhead{Designation} & \colhead{\kms} & \colhead{arcsec} & \colhead{1}}
\startdata
A & -69.07 & 3.75 & 0.05 \\ 
B & 167.17 & 5.99 & 0.00 \\ 
C & 7.01 & 9.59 & 0.00 \\ 
D & -74.27 & 9.72 & 0.06 \\ 
E & -38.64 & 10.29 & 0.02 \\ 
F & -10.51 & 10.33 & 0.01 \\ 
G & -94.29 & 12.34 & 0.10 \\ 
H I\tablenotemark{c} & 177.58 & 12.81 & 0.00 \\ 
H II\tablenotemark{c} & 23.92 & 12.81 & 0.00 \\ 
J & 39.04 & 15.46 & 0.00 \\ 
K & -155.96 & 16.30 & 0.36 \\ 
L & -88.29 & 16.92 & 0.09 \\ 
M & 14.61 & 17.20 & 0.00 \\ 
N & -81.88 & 19.05 & 0.07 \\ 
O & -10.71 & 21.37 & 0.01 \\ 
P & 86.69 & 21.60 & 0.00 \\ 
Q & -59.06 & 22.43 & 0.04 \\ 
R & 41.44 & 23.45 & 0.00 \\ 

\enddata
\tablenotetext{a}{We could not reliable measure a radial velocity for Candidate I.}
\tablenotetext{b}{Probability to be the donor star for the given \vrad\ and using the priors from \gls{besancon} and  \citet{2008ApJ...677L.109H}. We assume in this case that  one of nineteen candidates must be the donor star.}
\tablenotetext{c}{Extraction region H shows to different radial velocities, stemming from the two different stars visible in the HST image.}
\end{deluxetable}
